\chapter{Contexte}
    \section{Fonctionnenment d'une bibliothèque}
\paragraph{}
De façon générale, les prêts au sein d'une bibliothèque
exige la présence de grand nombre de ressources. En se 
basant directement sur la réalité liée à une entité
universitaire, le processus est le suivant: \par 
\begin{enumerate}
    \item Un étudiant désire faire un prêt.
    \item Il se rend à la bibliothèque.
    \item L'étudiant fait sa demande.
    \item Le responsable va vérifier si le livre demandé est disponible.
    \begin{itemize}
        \item[*] Si le livre est disponible, le responsable enregistre le prêt 
        et le donne au demandeur.
        \item[*] Si le livre n'est pas disponible, il l'en informe.
    \end{itemize} 
    \item L'étudiant s'en va avec ou sans livre.
\end{enumerate}

    \section{Affluence résultante}
\paragraph{}
Un tel processus pourrait avoir des conséquences négatives
sur le fonctionnement de la bibliothèque. Pour commencer, cela pourrait
être irritant pour un étudiant de faire un déplacement inutile
dans le but d'emprunter un livre et ne pas le trouver. De plus, 
il se pourrait que plusieurs étudiants aient besoin de divers livres
au même instant. Le comble pour le responsable sera de desservir tout 
un groupe. Non seulement l'affluence pourrait nuire au calme demandé
dans un tel espace, mais cela fatiguerait rapidement le bibliothécaire.
    
    \section{Avantages portées par la numérisation}
\paragraph{}
Grâce à la numérisation du système de gestion des prêts,
il deviendra possible pour un intéressé de faire des consultations
en ligne. Ceci diminuerait tant les déplacements inutiles que les 
affluences. En plus, certaines tâches pourraient facilement être 
automatisées, ce qui aiderait à limiter les erreurs humaines.   
\paragraph{}
Dans le soucis d'assurer la faciliter une eventuelle expansion du projet ou handover, nous laissons à la disposition de l'équipe en fontion la méthodologie utilisée pour la réalisation du dit projet.